\section{Inference}

To generate samples from the distribution represented by a probabilistic program $f$, we apply the Metropolis Hastings (MH) algorithm. Instead of taking the support of the distribution itself as the space $\mathbb{X}$ to explore, we explore the space of valid traces of $f$, since we can for any given valid trace $t$ evaluate its probability with $sdeval(f,t)$, whereas we can not do the same for some given value from the support of the distribution represented by $f$. However, since with every given trace we also get a corresponding return value, we indirectly do get samples from the actual distribution represented by $f$.

We define therefore $\mathbb{X} := T_{f,\text{valid}}$, the space of all valid probabilistic program traces for $f$, and $\tilde{\pi}(t) := sdeval(f,t)$, the semi-deterministic evaluation of $f$ for a given trace $t$ (implicitly taking $sdeval$ here to only be returning the calculated probability). Since we are restricting our space to only valid traces of $f$, the evaluation with $sdeval$ is fully deterministic. $\tilde{\pi}(t)$ is therefore non-zero for any valid trace $t$ of $f$ that does not determine an impossible value for any of the primitive distributions recorded in it and neither results in any observe statements in $f$ evaluating to a probability of $0$ during execution.

As the proposal kernel $q$ we could take any scheme that proposes a new trace $t'$ given a prior trace $t$, as long as we can evaluate the fraction $\frac{q(t_t | \hat{t}_{t+1})}{q(\hat{t}_{t+1} | t_t)}$ for it to calculate the MH acceptance ratio. We take here perhaps simplest choice for $q$, a kernel where we randomly pick one primitive entry in the current trace and pick a new value for it, leaving the rest of the trace as is. We do so "flat-uniformly", meaning that any primitive distribution appearing in the trace is equally likely, no matter where in the tree structure it is \cite{wingate2011lightweight}. Though different design choices could be made in this regard.

How we pick a new value $v'$ for some primitive entry $P(d,p,v)$ can also be done in any of many ways. We could simply draw a new sample from the distribution $d[p]$, independent from the prior value $v$. But we also could come up with a more informed local kernel $q_{d[p]}[v]$ for a primitive distribution, picking the new value in some way dependent on the prior value $v$. For example for a distribution $d[p]$ defined on $\mathbb{R}$, we could take as its local kernel a normal distribution centered around the prior value, $q_{d[p]}[v] := \mathcal{N}[\mu = v, \sigma^2]$ (for some appropriate choice of $\sigma^2$). For the sake of generality we will assume that for every primitive distribution $d[p]$ some local kernel $q_{d[p]}[v]$ is defined, which might or might not actually depend on $v$ and could just be the distribution $d[p]$ itself.

Formally, we define the procedure for the kernel $q[t]$ as:

\begin{minipage}{\linewidth}
\begin{itemize}
\item Flat-uniformly pick a primitive entry $P(d,p,v)$ in the trace $t$
\item Sample a proposal value $v' \sim q_{d[p]}[v]$
\item Define the intermediate proposal $\tilde{t}'$ as $t$ with $P(d,p,v)$ replaced by $P(d,p,v')$
\item Evaluate $sdeval(f,\tilde{t}')$ to get the valid proposal trace $t'$
\item Return $t'$
\end{itemize}
\end{minipage}

The probability of picking a particular entry $P(d,p,v)$ in $t$ is $\frac{1}{|t|}$ where $|t|$ is the number of primitive entries in $t$, and the probability of picking a particular new value $v'$ for $P(d,p,v)$ is $q_{d[p]}(v' | v)$. So the total probability of getting some particular intermediate proposal $\tilde{t}'$ is $\frac{1}{|t|} q_{d[p]}(v' | v)$.

Since $\tilde{t}'$ is possibly an invalid trace for $f$, $sdeval(f,\tilde{t}')$ might consequently be non-deterministic. So the probability of drawing the particular proposal trace $t' \sim q[t]$ also depends on whether any sample expressions were encountered during the execution of $sdeval(f, \tilde{t}')$ for which there was no determining entry in $\tilde{t}'$, and what values with what corresponding probability were drawn from the primitive distribution.

To compute $p(sdeval(f,\tilde{t}') = t')$ we iterate through $t'$ and compare it with $\tilde{t}'$, looking for entries in $t'$ that have no corresponding entry in $\tilde{t}'$. This can either be because the entry at the same position in the trace tree isn't a primitve of the same kind, or because there is no entry at all at the same position. In either case, any such entry must have been non-deterministically generated during the execution of $sdeval(f,\tilde{t}')$. The product of the individual probabilities of all such entries therefore gives us $p(sdeval(f,\tilde{t}') = t')$.

With that we have for our kernel the following probability density:

\begin{equation*}
     q(t' | t) = \frac{1}{|t|} q_{d[p]}(v' | v) p(sdeval(f,\tilde{t}') = t')
\end{equation*}

To be noted is that to calculate the reverse kernel probability $q(t_t | \hat{t}_{t+1})$ for the acceptance ratio, we don't actually have to run $sdeval(f,\cdot)$. Rather, we only have consider the hypothetical reverse intermediate proposal $\tilde{t}_t$, which is $\hat{t}_{t+1}$ with $P(d,p,\hat{v}_{t+1})$ once again replaced by $P(d,p,v_t)$, and compare it with $t_t$ as elaborated above to calculate $p(sdeval(f,\tilde{t}_t) = t_t)$.

So in total, for a flat-uniformly chosen primitive entry $P(d,p,v_t)$ in $t_t$ and proposal value $\hat{v}_{t+1} \sim q_{d[p]}[v_t]$, the acceptance ratio $\alpha$ for our MH algorithm is:

\begin{align*}
\alpha &= \frac{\tilde{\pi}(\hat{t}_{t+1})}
               {\tilde{\pi}(t_t)}
          \frac{q(t_t | \hat{t}_{t+1})}
               {q(\hat{t}_{t+1} | t_t)}\\
       &= \frac{sdeval(f, \hat{t}_{t+1})}
               {sdeval(f, t_t)}
          \frac{\frac{1}{|\hat{t}_{t+1}|} \; q_{d[p]}(v_t | \hat{v}_{t+1}) \; p(sdeval(f,\tilde{t}_t) = t_t)}
               {\frac{1}{|t_t|} \; q_{d[p]}(\hat{v}_{t+1} | v_t) \; p(sdeval(f,\tilde{t}_{t+1}) = \hat{t}_{t+1})}
\end{align*}

If we make sure to keep the result of $sdeval(f, t_t)$ stored between steps, this means that in total we have to only evaluate the possibly very expensive computation $sdeval(f, \cdot)$ once per MH iteration, that is to turn the invalid intermediate proposal trace $\tilde{t}_t$ into the valid proposal trace $\hat{t}_{t+1}$ (and with that to also get the value of $\tilde{\pi}(\hat{t}_{t+1})$).

With all prerequisites for MH provided, we can apply the algorithm and explore our trace space $\mathbb{X}$ to generate a Markov chain of traces $(t_t)_{t \in \mathbb{N}_0}$ of $f$ that converges to being representative of the distribution $\pi$ of valid traces of $f$ as defined by the semantics of our probabilistic program.

And since with the evaluation of $sdeval(f,t)$ for some trace $t$ we also the respective return value of the probabilistic program, we get with this also a sampling procedure for the actual distribution defined by $f$ as desired.

In total, the MH procedure to sample from the distribution defined by some probabilistic program $f$ looks as follows:

\begin{minipage}{\linewidth}
\begin{itemize}
\item Initialize our first trace $t_0 := sdeval(f,\emptyset)$\footnote{To ensure we only yield possible samples, we should resample $t_0$ until $\tilde{\pi}(t_0) > 0$.}
\item Repeat forever:
  \begin{itemize}
  \item Flat-uniformly pick a primitive entry $P(d,p,v)$ in the trace $t_t$
  \item Sample a proposal value $\hat{v}_{t+1} \sim q_{d[p]}[v_t]$
  \item Define $\tilde{t}'_{t+1}$ as $t_t$ but with $P(d,p,v_t)$ replaced by $P(d,p,\hat{v}_{t+1})$
  \item Evaluate $sdeval(f,\tilde{t}'_{t+1})$ to get the valid proposal trace $\hat{t}_{t+1}$
  \item Calculate the acceptance ration $\alpha$ as described above
  \item Sample $u \sim \mathcal{U}[0,1]$
  \item If $u < \alpha$ then $t_{t+1} := \hat{t}_{t+1}$, otherwise $t_{t+1} = t_t$
  \item Yield the return value associated with $t_{t+1}$ as a sample value
  \end{itemize}
\end{itemize}
\end{minipage}
