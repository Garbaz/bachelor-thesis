\section{Metropolis Hastings}

While there are many Markov chain Monte Carlo algorithms (MCMC), with differing conditions for application and advantages/disadvantages in terms of implementation complexity and performance, we will focus here on a principally rather simple algorithm, the "Metropolis-Hastings algorithm" (MH) \cite{metropolis1953equation} \cite{hastings1970monte}.

The problem we are trying to solve with MH is obtaining representative samples from some distribution of interest $\pi$, the \textit{target}, for which we only are able to compute the probability density $\pi(\cdot)$. And as is the principle of MCMC methods, we do so by exploring the space $\mathbb{X}$ of possible samples from $\pi$, its \textit{support}, via the iterative development of a Markov chain through this space. How exactly these steps are taken is the principal concern of MH.

To apply MH to sampling from some distribution $\pi$, we first need to pick a (usually, relative to $\pi$, very simple) distribution $q[x]$, the \textit{proposal kernel}, with the same support as $\pi$. This kernel defines our exploration of the support $\mathbb{X}$. Specifically, at every step $x_t$ in our Markov chain we use $q$ to choose a possible next step $\hat{x}_{t+1} \sim q[x_t]$. So $q$ can, and usually will, depend on the current position $x_t$ in $\mathbb{X}$.

If our $\pi$ were for example defined over $\mathbb{X} = \mathbb{R}^2$, a possible candidate for $q$ would be a two-dimensional normal distribution centered around $x_t$. So at each step in the Markov chain exploration of $\mathbb{X}$, we would draw $\hat{x}_{t+1} \sim \mathcal{N}^2[\mu = x_t, \sigma^2]$ (with $\sigma^2$ being a parameter to be tuned for faster convergence).

If we were to simply always take this proposed step $\hat{x}_{t+1}$ drawn from $q[x_t]$, then the result would be a random walk through $\mathbb{X}$ entirely independent from our target distribution $\pi$. With the goal of course being to obtain a series of samples from $\pi$, this would not be of much use.

The second part to every sampling step in the MH algorithm is to decide whether to take the proposed step drawn from $q[x_t]$, $x_{t+1} := \hat{x}_{t+1}$, or whether to remain at the current position in $\mathbb{X}$, $x_{t+1} := x_t$. This decision is itself done randomly:

\begin{align*}
x_{t+1} & = \begin{cases}
    \hat{x}_{t+1} & \text{ if } u \le \alpha \\
            x_{t} & \text{ otherwise }
            \end{cases} \\
\alpha & = \frac{\pi(\hat{x}_{t+1})}{\pi(x_t)} \frac{q(x_t | \hat{x}_{t+1})}{q(\hat{x}_{t+1} | x_t)} \\
u & \sim \mathcal{U}[0,1]
\end{align*}

The value $\alpha$ defined here is called the \textit{acceptance ratio}. If $\alpha$ is equal to $0$, say if $\pi(\hat{x}_{t+1}) = 0$, then we will definitely remain in place, since $P(U \le 0) = 0$. So our Markov chain will never step to values which are impossible to be a sample from $\pi$.

If on the other hand $\alpha$ is greater than or equal to $1$, say if $\pi(\hat{x}_{t+1}) \ge {\pi(x_t)}$ and $q(x_t | \hat{x}_{t+1}) = q(\hat{x}_{t+1} | x_t)$, then we will definitely take the proposed step, since $P(U \le 1) = 1$. So our Markov chain will always tend towards regions of more likely values under $\pi$.

For any $\alpha$ between $0$ and $1$, we will sometimes take the proposed step and sometimes will not, depending on what value is drawn for $\mathcal{U}[0,1]$. So we can still step "back down" to values that are less likely than the current value, but this is increasingly unlikely the smaller the ratio between the probability under $\pi$ of the value after a proposed step and of the current value. As a result, we will generally tend towards sampling from regions of high probability under $\pi$, while also in the long run exploring regions of lower (non-zero) probability.

And under some assumptions about the target distribution $\pi$ and the kernel $q$ it is possible to rigorously prove that, at least in the limit, the Markov chain of samples generated as such will eventually converge to being a sequence of (dependent) samples from $\pi$ \cite{metropolis1953equation}.

So in total the complete Metropolis-Hastings algorithm looks as follows:

\begin{minipage}{\linewidth}
\begin{itemize}
\item Repeat forever:
  \begin{itemize}
  \item Sample $\hat{x}_{t+1}$ from $q(x_t)$
  \item Calculate acceptance ratio $\alpha$ based upon $\hat{x}_{t+1}$ and $x_t$
  \item Sample $u \sim \mathcal{U}[0,1]$
  \item If $u \le \alpha$, then $x_{t+1} := \hat{x}_{t+1}$, else $x_{t+1} := x_t$
  \end{itemize}
\end{itemize}
\end{minipage}

One in practice often times highly relevant property to note about the definition of MH is the fact that we only ever need to compute a ratio $\frac{\pi(x)}{\pi(y)}$ between two results of the probability density function $\pi(x)$. This means that we do not actually have to be able to compute $\pi(x)$ directly, but rather that it is sufficient to be able to compute some proportional function $\tilde{\pi}(x) \propto \pi(x)$, since $\frac{\tilde{\pi}(x)}{\tilde{\pi}(y)} = \frac{\pi(x)}{\pi(y)}$.

As a very common practical example, say we would like to generate samples from some posterior distribution:

\begin{equation*}
\pi(x) = p(x | w) = \frac{p(w | x) p(x)}{p(w)}
\end{equation*}

While calculating $p(w | x)$ and $p(x)$ might be straightforward, often times directly getting a value for $p(w)$ is rather difficult. Usually one would have to compute it from the other two quantities as $p(w) = \int p(w | x) p(x) dx$.

With MH however, since $p(w)$ does not depend on $x$ and we only need to know $\pi(x)$ up to a proportionality constant, we can simply define $\tilde{\pi}(x) = p(w) \pi(x) = p(w | x) p(x)$ and calculate our acceptance ratio $\alpha$ with respect to $\tilde{\pi}$ rather than $\pi$, sidestepping the need to evaluate any possibly intractable integrals.
